\chapter*{Introduction\markboth{Introduction}{}}
\addcontentsline{toc}{chapter}{Introduction}

The Datagram Transport Layer Security (DTLS) provides security on top of datagrams protocols such as UDP. It was originally presented as an adaptation of TLS \cite{rfc5246} for unreliable communications by Modadugu \& Rescorla in 2004 \cite{modadugu2004design}. This new standard arised whereas some software developers already had built their own way of dealing with secure communications over UDP. However, the interest for this protocol is increasing and its integration in existing systems when security is needed is more and more considered \cite{dtls-as-subtransport}.

The main advantage of this protocol is its tolerance to frequent losses, which could be a requirement in some environment. This may also be helpful for real-time communication such as live streaming, where retransmission is not required. We will present later on all the potential applications that could benefit from using DTLS. Moreover, some existing commercial applications are already actively using it, like Cisco AnyConnect VPN\cite{anyconnect}.


Our objective in this master thesis is to assess the possibility to bring multipath ability to this protocol. Namely, we want to make it use multiple interfaces concurrently for the same DTLS session. This aspect have already been studied for other protocols such as MPTCP \cite{RFC6824} or MPRTP \cite{singh-avtcore-mprtp}. Nevertheless,we must keep in mind the additional security requirement and guarantee at least the same level of security as standard DTLS.

In the first chapter of this master thesis, we present additional details about DTLS with a short summary of the foundations inherited from TLS. We give an example of how a typical handshake will take place and how security is guaranteed. Then, we present an overview of application data exchange and how losses are handled. Finally, we review the security considerations of DTLS.

Chapter \ref{chap:mprtp} is giving an insight into an existing multipath protocol, MPRTP. This is probably the closest protocol we have to be inspired of. Indeed, it is most of time used with UDP and has also to deal with losses and reordering. This chapter concludes the first part dedicated to the state of the art.

The following chapters constitute the second part of this thesis and focus on our proposition : MPDTLS. Chapter \ref{chap:design} details the different design aspects we have investigated, the various mechanisms built to make the multipath possible such as addresses advertisements and flow establishment. In addition, we review all the new packets introduced to support those mechanisms.

Chapter \ref{chap:implementation} describes our work on the implementation of the design in an existing DTLS library. We show how the library works and the typical data flow. Then, we expose our data structures and processes to support multipath.

Finally, Chapter \ref{chap:perfs} presents concrete results we have obtained with our implementation and a simple VPN application. In this part, we show that our design and the subsequent implementation are giving good results and really take advantage of multiple interfaces.

